\documentclass[landscape]{slides}
\usepackage[landscape, margin=2cm]{geometry}
\usepackage{color}
\usepackage{bm}
\usepackage{graphicx}
\usepackage{hyperref}
\usepackage{listings}
\definecolor{mygreen}{rgb}{0,0.6,0}
\definecolor{mymauve}{rgb}{0.58,0,0.82}
\lstset{ %
  backgroundcolor=\color{white},   % choose the background color; you must add \usepackage{color} or \usepackage{xcolor}
  basicstyle=\footnotesize,        % the size of the fonts that are used for the code
  commentstyle=\color{mygreen},    % comment style
  deletekeywords={...},            % if you want to delete keywords from the given language
  extendedchars=true,              % lets you use non-ASCII characters; for 8-bits encodings only, does not work with UTF-8
  frame=single,                    % adds a frame around the code
  keywordstyle=\color{blue},       % keyword style
  language=Python,                 % the language of the code
  rulecolor=\color{black},         % if not set, the frame-color may be changed on line-breaks within not-black text (e.g. comments (green here))
  showspaces=false,                % show spaces everywhere adding particular underscores; it overrides 'showstringspaces'
  stringstyle=\color{mymauve},     % string literal style
  title=\lstname                   % show the filename of files included with \lstinputlisting; also try caption instead of title
}
\hypersetup{
    colorlinks=true,
    urlcolor=blue
}

\graphicspath{ {./img/} {./charts/} }


\title{Factory Boy Fun}
\author{Adam Johnson - me@adamj.eu}
\date{9th September 2014}

\begin{document}

\maketitle

\begin{slide}

    \textcolor{blue}{\Large{What the problem?}}

    \begin{itemize}
        \item You've done this, right?
    \end{itemize}

    \begin{lstlisting}
# tests/test_something.py
class MyTests(TestCase):
    fixtures = ['basic.json']
    def setUp(self):
        self.user = User.objects.create(
            username='adam',
            first_name='Adam',
            last_name='Johnson',
            email='adam@example.com'
        )
    \end{lstlisting}

\end{slide}


\begin{slide}
    \begin{lstlisting}
# tests/test_something.py
class MyTests(TestCase):
    fixtures = ['basic.json']
    def setUp(self):
        self.user = User.objects.create(
            username='adam',
            first_name='Adam',
            last_name='Johnson',
            email='adam@example.com'
        )
    \end{lstlisting}

    \textbf{Pain...}

    \begin{itemize}
        \item Test data in two places!
        \item A quirky json file that has to be maintained separately!
        \item I \emph{“just”} want a User but I have to give \emph{every} detail!
    \end{itemize}
\end{slide}


\begin{slide}
    \textcolor{blue}{\Large{Model-building can overtake testing}}

    \begin{itemize}
        \item Let's fix that... with a package ported from Ruby on Rails. (Trust me, it's gonna be okay!)
    \end{itemize}
\end{slide}

\begin{slide}

    \textcolor{blue}{\Large{Factory Boy}}

    \begin{itemize}
        \item “a fixtures replacement based on thoughtbot’s `factory\_girl'.”
        \item \url{http://factoryboy.readthedocs.org/en/latest/}
    \end{itemize}

\end{slide}



\begin{slide}

    \textcolor{blue}{\Large{Example factory}}

    \begin{itemize}
        \item Doesn't fit on one slide... here's the imports:
    \end{itemize}

    \begin{lstlisting}
# app/factories.py
from datetime import datetime, timedelta
from random import randint
from django.template.defaultfilters import slugify
from factory import DjangoModelFactory, lazy_attribute

now = datetime.now
    \end{lstlisting}

\end{slide}




\begin{slide}

    \begin{lstlisting}
class User(DjangoModelFactory):
    class Meta:
        model = 'auth.User'
        django_get_or_create = ('username',)

    first_name = 'Adam'
    last_name = 'Johnson'

    @lazy_attribute
    def username(o):
        return slugify(o.first_name + '.' +
                       o.last_name))

    @lazy_attribute
    def email(self):
        return self.username + "@example.com")

    @lazy_attribute
    def date_joined(self):
        return now() - timedelta(days=randint(5, 50))

    @lazy_attribute
    def last_login(self):
        return self.date_joined + dt.timedelta(days=4))
    \end{lstlisting}

\end{slide}


\end{document}
